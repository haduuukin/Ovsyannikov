\documentclass{beamer}
\usetheme{Singapore}
\usepackage[T1,T2A]{fontenc}
\usepackage[cp1251]{inputenc}
\usepackage[russian]{babel}
\title{Тестовые слайды}
\subtitle{Основы информационной безопасности}
\author{Овсянников Никита Владиславович, 2КБ}
\date{\today}
\institute{БФУ им. И. Канта}

\begin{document}

\begin{frame}
\titlepage
\end{frame}

\begin{frame}
\frametitle{Описание курсовой работы}
\begin{center}

\textbf{Тема курсовой:}\\
«Безопасность мобильных мессенджеров»\\

\underline{\textit{Задача курсовой:}}\\
Рассмотреть основные угрозы безопасности мобильных мессенджеров, изучить типы угроз, возможные утечки данных пользователей и имеющиеся методы защиты\\

\underline{\textit{Методы исследования:}}\\
Изучение и анализ источников в интернет. Для анализа были взяты 3 самых популярных мессенджера в России: WhatsApp, Viber, Telegram. Были рассмотрены методы шифрования данных, аутефикция пользователей и был произведён анализ утечек и крупных взмломов мессенджеров.

\end{center}
\end{frame}

\begin{frame}
\frametitle{Результаты курсовой работы}
Были рассмотрены различные типы угроз и методы защиты от них. Большинство мессенджеров для защиты данных пользователя, используют метод сквозного шифрования, но за последние годы было совершенно большое количество успешных атак и утечек данных пользователей, из чего можно сделать вывод, что большинство мессенджеров небезопасны и имеют множество проблем с защитой данных.
\end{frame}

\end{document}

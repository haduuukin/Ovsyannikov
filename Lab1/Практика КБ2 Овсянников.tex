\documentclass[a4paper,12pt]{article}
\usepackage{amsmath,amsthm,amssymb}
\usepackage{mathtext}
\usepackage[top=2cm, bottom=2cm]{geometry}
\usepackage[T1,T2A]{fontenc}
\usepackage[cp1251]{inputenc}
\usepackage[russian]{babel}
\newtheorem{theorem}{Теорема}
\newtheorem{corollary}{Следствие}[theorem]
\newtheorem{lemma}[theorem]{Лемма}
\newtheorem*{remark}{Пример}
\usepackage{fancyhdr}
\pagestyle{fancy}
\fancyhf{}
\renewcommand{\headrulewidth}{0pt}
\rhead{\textbf{\thepage}}
\setcounter{page}{307}
\chead{\textbf{Конечные непрерывные дроби}}
\lhead{\textbf{Раздел 13-3}}

\begin{document}
	Это говорит о том, что конвергентами $[0; 2, 1, 2, 6]$ являются
	\begin{equation} \begin{split}
		C_0=p_0/q_0=0, C_1=p_1/q_1=1/2, C_2=p_2/q_2=1/3, C_3=p_3/q_3=3/8,\\ C_4=p_4/q_4=19/51,
	\end{split} \end{equation}
	как мы знаем, что они должны быть.

\indent{Мы продолжаем наше освоение свойств конвергентов путем доказательства}
	\begin{theorem} \label{1} Если $C_k=p_k/q_k$ это $k$-ый конвергент простой непрерывной дроби $[a_0; a_1, ...,a_n]$, тогда \[p_k q_{k-1}-q_k p_{k-1}=(-1)^{k-1}, 1\leq k \leq n.\] \end{theorem}
    \begin{proof} Индукция на $k$ работает довольно просто, с соотношением
    \[p_1 q_0 - q_1 p_0 = (a_1 a_0 + 1) \cdot 1 - a_1 \cdot a_0 = 1 = (1)^{1-1},\]\\
    избавившись от значения $k=1$. Мы предполагаем, что рассматриваемая формула
    также справедлива для  $k=m$, где $0\leq m \leq n.$ Тогда
    \[p_{m+1} q_m - q_{m+1} p_m = (a_{m+1} p_m + p_{m-1}) q_m - (a_{m+1} q_m + q_{m-1}) p_m\] \[= -(p_m q_{m-1} - q_m p_{m-1})\] \[=-(-1)^{m-1} = (-1)^m \]
    и поэтому формула справедлива для $m+1$, когда она справедлива для $m$. Это следует из индукции, что она действительна для всех $k$ при $1\leq k \leq n$ \end{proof}
    \indent Заметным следствием этого результата является то, что числитель и знаменатель любого конвергента относительно просты, так что конвергенты всегда задаются на самых низких членах.\\
    \begin{corollary} Если $d=\text{НОД}(p_k, q_k)$, тогда, следуя из теоремы, $d|(-1)^{k-1}$; пока $d>0$, это заставляет нас заключить, что $d=1$.\end{corollary}
    \begin{remark}\end{remark}
    Рассмотрим непрерывную дробь $[0; 1, 1, ..., 1]$, в которой все частные знаменатели равны 1. Здесь первые несколько конвергентов являются 
    \[C_0 = 0/1, C_1 = 1/1, C_2 = 2/1, C_3 = 3/2, C_4 = 5/3, \ldots\]
    Так как числитель $k$-го сходящегося $C_k$ равен
    \[p_k = 1\cdot p_{k-1} + p_{k-2} = p_{k-1}+ p_{k-2} \]
    
    \newpage
    \renewcommand{\headrulewidth}{0pt}
    \lhead{\textbf{\thepage}}
    \setcounter{page}{308}
    \chead{\textbf{Числа Фибоначчи и непрерывные дроби}}
    \rhead{\textbf{Глава 13}}
     \noindent {а знаменатель равен}\\
    \[q_k = 1\cdot q_{k-1} + q_{k-2} = q_{k-1}+ q_{k-2} \]
    очевидно, что
    \[C_k = u_{k+1/ u_k} (k\geq 2),\]
  где $u_k$ обозначает $k$-е число Фибоначчи. В данном контексте,  тождество $p_k q_{k-1} - q_k p_{k-1} = (-1)^{k-1}$ Теоремы 13-7 принимает вид \[u_{k+1} u_{k-1} - u^2_k = (-1)^{k-1};\]
  это именно формула (3) на стр. 294.\\\\
  \indent {Обратимся теперь к линейным диофантовым уравнениям} 
  \[a x + b y = c,\]
  где $a, b, c$ это целые числа. Поскольку никакого решения этого уравнения не существует, если $d \nmid c$, где $d=\text{НОД}(a, b)$, нет ничего плохого в предположении, что $d \mid c$. На самом деле, нам нужно только позаботиться о ситуации, в которой коэффициенты относительно просты. Если $\text{НОД}(a, b)=d>1$, тогда уравнение можно разделить на $d$, чтобы получить 
  \[(a/d)x + (b/d)y = c/d\]
  Оба уравнения имеют одинаковое решение, и в последнем случае мы знаем, что $\text{НОД}(a/d, b/d) = 1.$\\
  \indent Заметим также, что решение уравнения 
  \[a x + b y = c, \text{НОД}(a, b) = 1\]
  может быть получен путем первого решения диофантова уравнения
  \[a x + b y = 1, \text{НОД}(a, b) = 1.\]
  Действительно, если можно найти целые числа $x_0$ и $y_o$, для которых\\ 
  $a x_0 + b y_0 = 1$, то умножение обеих сторон на $c$ даёт
  \[a(c x_0)  + b(c y_0) = c\]
  Следовательно, $x=c x_0$ и $y=c y_0$ это искомое решение $a x + b y = c.$\\
  \indent Чтобы получить пару целых чисел $x$ и $y$, удовлетворяющих уравнению $a x + b y = 1$, разверните рациональное число $a/b$  как простую непрерывную дробь; скажем,
  \[a/b = [a_0; a_1, \ldots , a_n]\]
 \newpage
 \renewcommand{\headrulewidth}{0pt}
 \rhead{\textbf{\thepage}}
 \setcounter{page}{309}
 \chead{\textbf{Бесконечные непрерывные дроби}}
 \lhead{\textbf{Раздел 13-3}}
 \noindent Теперь последние два конвергента этой непрерывной дроби это
 \[C_{n-1}=p_{n-1}/q_{n-1} \text{ и } C_n=p_n/q_n=a/b\]
 Поскольку $\text{НОД}(p_n, q_n)=1=\text{НОД}(a, b)$, можно сделать вывод, что
 	\[p_n=a \text{ и } q_n=b.\]
 В силу теоремы \ref{1}, мы имеем
 \[p_nq_{n-1} - q_np_{n-1} = (-1)^{n-1}\]
 или, с изменением обозначения 
 \[a q_{n-1} - b p_{n-1} = (-1)^{n-1}\]
 Таким образом, с $x = q_{n-1}$ и $y = - p_{n-1}$, мы имеем
 \[a x + b y = (-1)^{n-1}\]
 если $n$ нечетно, то уравнение $a x + b y = 1$ имеет частное решение 
 $x_0 = q_{n-1} $, $y_0 = -p_{n-1}$, в то время как если n это четное целое число, то решение дается следующим образом: $x_0 = -q_{n-1}$, $y_o = p_{n-1}$. Наши ранние теории говорят нам, что общее решение это 
 	 \[x = x_0 + b t, y = y_0 - a t,  (t=0, \pm1,\pm2, \ldots).\]
 	 \begin{remark}\end{remark}
 	 Решим линейное диофантово уравнение
 	 \[172x + 20y = 1000\]
 	 с помощью простой непрерывной дроби. Начиная c $\text{НОД}(172, 20) = 4$, это уравнение может быть заменено уравнением
 	 \[43 x + 5 y = 250\]
 	 первый шаг это найти частное решение
 	 \[43 x + 5 y = 1\]
 	 Чтобы достичь этого, мы начнем с записи $43/5$ (или, если кто-то предпочитает, $5/43$) в виде простой непрерывной дроби. Последовательность равенств, полученная путем применения Евклидова алгоритма к числу $43$ и $5$, имеет вид
 	 \begin{equation} \begin{split}
 	 	43 = 8\cdotp5 + 3,\\
 	 	5 = 1\cdotp3 + 2,\\
 	 	3=1\cdotp2 + 1,\\
 	 	2=2\cdotp1,
 	 \end{split} \end{equation}
 	 \newpage
 	 \renewcommand{\headrulewidth}{0pt}
 	 \lhead{\textbf{\thepage}}
 	 \setcounter{page}{310}
 	 \chead{\textbf{Числа Фибоначчи и непрерывные дроби}}
 	 \rhead{\textbf{Глава 13}}
Итак $43/5 = [8;1, 1, 2] = 8 + \dfrac{1}{1+\dfrac{1}{1 + \frac{1}{2}}}$. Конвергентами этой непрерывной дроби являются
\[C_0 = 8/1, C_1 = 9/1, C_2 = 17/2, C_3 = 43/5,\]
из чего следует, что $p_2 = 17, q_2 = 2, p-3 = 43$ и $q_3 = 5$. Возвращаясь снова к теореме \ref{1}
\[p_3 q_2 - q_3 p_2 = (-1)^{3-1}\]
или в эквивалентном термине,
\[43 \cdotp 2 - 5\cdotp17 = 1\]
Если это отношение умножить на $250$, то получим
\[43\cdotp500 + 5(-4250) = 250.\]
Таким образом, частное решение диофантова уравнения $43 x +5 y =250$
\[x_0 = 500, y_0 = -4250\]
Общее решение дается уравнениями
\[x = 500 + 5 t , y = -4250 - 43 t, (t = 0, \pm1, \pm2 \ldots).\]
\indent Прежде чем доказывать теорему о поведении нечетных и четных конвергентов простой непрерывной дроби, требуется предварительная лемма.\\
\begin{lemma} Если $q_k$ это знаменатель $k$-го сходящегося конвергента $C_k$ простой непрерывной дроби $[a_0; a_1, \ldots , a_n]$, то $q_{k-1} \leq q_k$ для $1\leq k\leq n$, со строгим неравенством, когда $k>1$. \end{lemma}
\begin{proof} Докажем лемму через индукцию. Во-первых, $q_0=1\leq a 1=q1$, так что утверждаемое равенство выполняется при $k=1$. Предположим, что это верно для $k=m$, где $1\leq m < n$. Тогда 
\[q_{m+1} = a_{m+1} q_m + q_{m-1} > a_{m+1} q_m \geq 1\cdotp q_m = q_m\]
Значит это неравенство также справедливо для $k = m+1.$ \end{proof}
\indent Имея в наличии такую информацию, легко доказать\\\\
\begin{theorem} Конвергенты с четными индексами образуют строго возрастающую последовательность, т. е.
\[C_0  < C_2 < C_4 < \cdots\ldotp\] \end{theorem}

\end{document}
